\documentclass[12pt,3p]{elsarticle}
\usepackage{tikz}
\usepackage{hyperref}
\usetikzlibrary{positioning,chains,fit,shapes,calc}
\newcommand{\swastik}[1]{%
    \begin{tikzpicture}[#1]
        \draw (-1,1)  -- (-1,0) -- (1,0) -- (1,-1);
        \draw (-1,-1) -- (0,-1) -- (0,1) -- (1,1);
    \end{tikzpicture}%
}

\tikzset{
    treenode/.style={align=center, inner sep=0pt},
    % Receiver Node
    node_rec/.style = {treenode, circle, white, font=\bfseries, draw=black, fill=black, text width=0.8cm},
    % Sender Node
    node_sen/.style = {treenode, circle, red, draw=red, very thick, text width=0.8cm}
}
\usepackage{graphicx}
\usepackage{amssymb}
\usepackage{minted}
\makeatletter
\def\ps@pprintTitle{%
 \let\@oddhead\@empty
 \let\@evenhead\@empty
 \def\@oddfoot{}%
 \let\@evenfoot\@oddfoot}
\makeatother
\setminted{breaklines=true, tabsize=2, breaksymbolleft=}
\begin{document}

\begin{frontmatter}
\title{Matrix Multiplication Optimisation Assignment}
\author{Sourabh Aggarwal}
\address{111601025@smail.iitpkd.ac.in}
\end{frontmatter}

\section{Optimisations}
\begin{enumerate}
  \item I started with basic blocking. Following are the results which I obtained which was expected as \texttt{LEVEL1\_DCACHE\_LINESIZE = 64}, this can be obtained by executing from terminal:
\begin{minted}{bash}
>> getconf LEVEL1_DCACHE_LINESIZE
\end{minted}
And since size of double is 8, optimality is obtained at block size of 8 (= 64/8).

\begin{minted}{bash}
  ------------ Results -------------
  Iteration: 1
  1, 6445.611429
  Iteration: 2
  2, 2272.318962
  Iteration: 3
  4, 897.756441
  Iteration: 4
  8, 725.708793
  Iteration: 5
  16, 731.699237
  Iteration: 6
  32, 906.470085
  Iteration: 7
  64, 999.502892
  Iteration: 8
  128, 975.474192
  Iteration: 9
  256, 1356.794308
  Iteration: 10
  512, 2576.451370
  Iteration: 11
  1024, 3304.505859
  Iteration: 12
  2048, 3478.298679
  Iteration: 13
  4096, 3459.928579
  ---------- END ----------
\end{minted}

\item After this I tried loop re-ordering and found the best ordering to be \texttt{i-k-j} (both for outer 3 loops and inner 3 loops). 

\item Thirdly I tried to save repeated pointer arithmetic by storing the common calculation in a variable, significant time was saved doing this. 

\item Lastly loop unrolling helped a lot. I started with unrolling the inner most loop which improved the time, then the second inner most which again improved the time but further unrolling increased the time taken.

\end{enumerate}

\section{Final Code}

Following code (final implementation) took around 170 seconds.

\begin{minted}{c}
#include <stdio.h>
#include <assert.h>
#include <time.h>

#define N 5120

double arr1[N][N], arr2[N][N], res[N][N];

int main() {
  // int B = CLS / sizeof(double); // CLS (Cache line size) = 64 in this machine and can be obtained using "getconf LEVEL1_DCACHE_LINESIZE".
  int B = 8;
  for (int i = 0; i < N; i++)
    for (int j = 0; j < N; j++) {
      arr1[i][j] = i + j + 0.5;
      arr2[i][j] = i + j + 0.5;
    }
  double *rres, *rmul1, *rmul2, tmp;
  int i, j, k, ii;
  double st = (double)clock();
  for (i = 0; i < N; i += B)
    for (k = 0; k < N; k += B)
      for (j = 0; j < N; j += B)
        for (ii = 0, rres = &res[i][j], rmul1 = &arr1[i][k]; ii < B; ii++, rres += N, rmul1 += N) {
          tmp = rmul1[0];
          rmul2 = &arr2[k][j];
          rres[0] += tmp * rmul2[0];
          rres[1] += tmp * rmul2[1];
          rres[2] += tmp * rmul2[2];
          rres[3] += tmp * rmul2[3];
          rres[4] += tmp * rmul2[4];
          rres[5] += tmp * rmul2[5];
          rres[6] += tmp * rmul2[6];
          rres[7] += tmp * rmul2[7];
          rmul2 += N;
          tmp = rmul1[1];
          rres[0] += tmp * rmul2[0];
          rres[1] += tmp * rmul2[1];
          rres[2] += tmp * rmul2[2];
          rres[3] += tmp * rmul2[3];
          rres[4] += tmp * rmul2[4];
          rres[5] += tmp * rmul2[5];
          rres[6] += tmp * rmul2[6];
          rres[7] += tmp * rmul2[7];
          rmul2 += N;
          tmp = rmul1[2];
          rres[0] += tmp * rmul2[0];
          rres[1] += tmp * rmul2[1];
          rres[2] += tmp * rmul2[2];
          rres[3] += tmp * rmul2[3];
          rres[4] += tmp * rmul2[4];
          rres[5] += tmp * rmul2[5];
          rres[6] += tmp * rmul2[6];
          rres[7] += tmp * rmul2[7];
          rmul2 += N;
          tmp = rmul1[3];
          rres[0] += tmp * rmul2[0];
          rres[1] += tmp * rmul2[1];
          rres[2] += tmp * rmul2[2];
          rres[3] += tmp * rmul2[3];
          rres[4] += tmp * rmul2[4];
          rres[5] += tmp * rmul2[5];
          rres[6] += tmp * rmul2[6];
          rres[7] += tmp * rmul2[7];
          rmul2 += N;
          tmp = rmul1[4];
          rres[0] += tmp * rmul2[0];
          rres[1] += tmp * rmul2[1];
          rres[2] += tmp * rmul2[2];
          rres[3] += tmp * rmul2[3];
          rres[4] += tmp * rmul2[4];
          rres[5] += tmp * rmul2[5];
          rres[6] += tmp * rmul2[6];
          rres[7] += tmp * rmul2[7];
          rmul2 += N;
          tmp = rmul1[5];
          rres[0] += tmp * rmul2[0];
          rres[1] += tmp * rmul2[1];
          rres[2] += tmp * rmul2[2];
          rres[3] += tmp * rmul2[3];
          rres[4] += tmp * rmul2[4];
          rres[5] += tmp * rmul2[5];
          rres[6] += tmp * rmul2[6];
          rres[7] += tmp * rmul2[7];
          rmul2 += N;
          tmp = rmul1[6];
          rres[0] += tmp * rmul2[0];
          rres[1] += tmp * rmul2[1];
          rres[2] += tmp * rmul2[2];
          rres[3] += tmp * rmul2[3];
          rres[4] += tmp * rmul2[4];
          rres[5] += tmp * rmul2[5];
          rres[6] += tmp * rmul2[6];
          rres[7] += tmp * rmul2[7];
          rmul2 += N;
          tmp = rmul1[7];
          rres[0] += tmp * rmul2[0];
          rres[1] += tmp * rmul2[1];
          rres[2] += tmp * rmul2[2];
          rres[3] += tmp * rmul2[3];
          rres[4] += tmp * rmul2[4];
          rres[5] += tmp * rmul2[5];
          rres[6] += tmp * rmul2[6];
          rres[7] += tmp * rmul2[7];
        }
  double en = (double)clock();       
  printf("time taken: %f", (en - st)/CLOCKS_PER_SEC);
  return 0;
}
\end{minted}

\section{Reference}

Besides basic material, I referred to \href{https://people.freebsd.org/~lstewart/articles/cpumemory.pdf}{this} text. 

\end{document}