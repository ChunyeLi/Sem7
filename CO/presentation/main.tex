\documentclass{beamer}

\usepackage{xcolor}  
\usepackage{environ}
\usepackage{minted}
\usepackage{tikzsymbols}

% map defaulf block to oldblock
\let\oldblock\block
\let\endoldblock\endblock
\providecommand{\tightlist}{%
  \setlength{\itemsep}{0pt}\setlength{\parskip}{0pt}}
% change block by adding smallskip
\renewenvironment{block}[1]
    {\begin{oldblock}{#1}
        \smallskip
    }
    { 
    \end{oldblock}
    }

\usetheme[progressbar=frametitle]{metropolis}

% Custom frame font begin
\newcommand{\customframefont}[1]{
\setbeamertemplate{itemize/enumerate body begin}{#1}
\setbeamertemplate{itemize/enumerate subbody begin}{#1}
}

\NewEnviron{framefont}[1]{
\customframefont{#1} % for itemize/enumerate
{#1 % For the text outside itemize/enumerate
\BODY
}
\customframefont{\normalsize}
}
% end

\makeatletter
\setlength{\metropolis@titleseparator@linewidth}{2pt}
\setlength{\metropolis@progressonsectionpage@linewidth}{2pt}
\setlength{\metropolis@progressinheadfoot@linewidth}{2pt}
\makeatother

\newcommand{\iph}[2]{
    \includegraphics[width=#1\textwidth,height=#1\textheight,keepaspectratio]{#2}
}
\newcommand{\ph}[1]{
    \includegraphics[width=0.5\textwidth,height=0.5\textheight,keepaspectratio]{#1}
}

\title{Static Links and Register Allocation}

\author{Sourabh Aggarwal}

\institute[IIT Palakkad] % (optional, but mostly needed)
{
  Department of Computer Science And Engineering\\
  IIT Palakkad
}

\date{1 November 2019}

\setminted{breaklines=true, tabsize=2, breaksymbolleft=, fontsize=\footnotesize}

\begin{document}

\begin{frame}
  \titlepage
\end{frame}


\begin{frame}[fragile]
  \frametitle{Escaping of Variables}
  \begin{itemize}

  \item Some variables must go into memory and not in register. 
  \pause
  \item If a variable is accessed in a nested function then that variable escapes.

  \end{itemize}
\end{frame}

\begin{frame}[fragile]
  \frametitle{Static Links}
  \begin{columns}
    \begin{column}{0.5\textwidth}
      \iph{1.15}{sl1}
    \end{column}
    \begin{column}{0.5\textwidth}
      Code demonstration
    \end{column}
    \end{columns}
  % frame pointer
\end{frame}

\begin{frame}[fragile]
  \frametitle{Liveness Analysis}
  \begin{itemize}
    \item Two temporaries \texttt{a} and \texttt{b} can fit into the same register, if \texttt{a} and \texttt{b} are never "in use" at the same time.
    \item We say a variable is live if it holds a value that may be needed in the future, so this analysis is called liveness analysis. To perform analyses on a program, we make a control-flow graph. Each statement in the program is a node in the flow graph; if statement \texttt{x} can be followed by statement \texttt{y} there is an edge from \texttt{x} to \texttt{y}.
    \item Liveness of node is calculated as shown:
    $in[n] = use[n] \cup (out[n] \setminus def[n])$

    $out[n] = \cup_{s \in succ[n]}in[s]$
  \end{itemize}
\end{frame}


\begin{frame}[fragile]
  \frametitle{Liveness Analysis}
  \begin{itemize}
    \item A condition that prevents \texttt{a} and \texttt{b} being allocated to the same register is called an interference.
    
    \item The most common kind of interference is caused by overlapping live ranges when \texttt{a} and \texttt{b} are both live at the same program point, then they cannot be put in the same register.

    \item Interference graph is created with vertices as temporaries and edges as follows:-
    \begin{itemize}
      \item At any nonmove instruction that defines a variable $a$, where the live-out variables are $b_1, …, b_j$, add interference edges $(a, b_1), …,(a, b_j)$.
      \item At a move instruction $a \leftarrow c$, where variables $b_1, …, b_j$ are live-out, add interference edges $(a, b_1), …, (a, b_j)$ for any $b_i$ that is not the same as $c$.
    \end{itemize}
  \end{itemize}
\end{frame}

\begin{frame}[fragile]
  \frametitle{Register Allocation}
  \begin{itemize}
    \item Now that we have made our interference graph, where each edge denotes that its endpoints must go in different registers, thus, our problem reduced into coloring our graph with K (\# no. of available registers) colors such that no two end points have same colour.
  \end{itemize}
\end{frame}

\begin{framefont}{\tiny}
  \begin{frame}
    \frametitle{Register Allocation}
    Recursive algorithm is defined as below, assume that we are given a set of instructions each of which may contain temporaries which might have been already mapped to some register.
    \begin{itemize}
    \item
      Construct interference graph given the list of instructions.
    \item
      Start with vertices (temporaries) which have not been assigned a
      colour (register).
    \item
      Such vertices whose \# Neighbours \(<\) K can be removed from the
      graph as we would always have a color available for them. Thus this
      divides our list of vertices into two viz., (simplify,
      potentialSpill).
    \item
      Now we should process these nodes in simplify, let ``stack'' denote
      the set of simplified nodes.
    \item
      Repeat until potentialSpill nodes become empty
      \begin{itemize}
      \item
        We ``simplify'' each node in our simplify list by removing it from
        simplify list and adding it to stack and decrementing the degree of
        its neighbors which are not in stack (stack nodes are already
        removed) and also which aren't already colored. Note that when we
        decrement the degree of the neighbors, if degree becomes less then K
        that means we can add this node to simplify and remove it from
        potentialSpill.
      \item
        If potentialSpill is empty then done, o/w select a node to spill,
        choose that which has maximum neighbors and should have accessed
        temps/memory least number of times. Assign it to simplify and remove
        it from potentialSpill.
      \end{itemize}
    \item
      Now process stack from top to bottom. For an element, colors available
      to it are total minus those taken by its neighbors, if colors are
      available then assign else we add it to actualSpill.
    \item
      If by now actualSpill is empty then we stop else we rewrite the
      program and repeat.
    \end{itemize}
  \end{frame}
\end{framefont}
\begin{frame}[standout]
Thanks!
\end{frame}   
\end{document}
